\documentclass[12pt,a4paper]{article}
\usepackage{graphicx} % Required for inserting images
\usepackage{multirow}
\usepackage{array}
\usepackage{hyperref}
\usepackage{float}
\usepackage{titlesec}
\usepackage{listings}
\usepackage{xcolor}
\usepackage{tabularx}
\usepackage{booktabs}


\lstset{
    language=Python,
    basicstyle=\ttfamily\small,
    breaklines=true,
    keywordstyle=\color{blue},
    identifierstyle=\color{black},
    commentstyle=\color{green},
    stringstyle=\color{red},
    numbers=left,
    stepnumber=1,
    tabsize=4,
}


\newcommand{\sectionbreak}{\clearpage}



\title{Python}
\author{Daniel Kubíček}
\date{September 2023}

\begin{document}
\maketitle
\tableofcontents
\section{Co je to Python a jeho vlastnosti}
\begin{itemize}
\normalsize
    \item Python je vysokoúrovňový programovací jazyk, který byl vyvinut Guido van Rossumem a poprvé uveden na trh v roce 1991. Jedná se o interpretovaný jazyk, což znamená, že kód může být spouštěn přímo bez předchozí kompilace. Python je známý svou jednoduchou a čitelnou syntaxí, což z něj činí oblíbený jazyk pro začínající programátory i profesionály.
    \item Některé klíčové vlastnosti Pythonu zahrnují:
     \item Čitelná syntaxe: Python používá odsazení (mezer nebo tabulátorů) k určení bloků kódu, což vede k čitelnému a konzistentnímu kódu.
     \item Vysokoúrovňový jazyk: Python má zabudované datové typy a funkce, což usnadňuje práci s různými úkoly bez potřeby složitých manipulací s pamětí.
     \item Multiplatformní: Python je dostupný na různých operačních systémech, což umožňuje psát kód, který běží na různých platformách.
     \item Rozšiřitelnost: Python umožňuje integraci kódu napsaného v jiných jazycích, což umožňuje využít existující knihovny a rozšíření.
     \item Velká komunita: Python má aktivní a rozsáhlou komunitu uživatelů a vývojářů, což znamená, že existuje mnoho knihoven a nástrojů pro různé účely.
     \item Python se používá pro širokou škálu aplikací, včetně webových aplikací, automatizace úkolů, datové analýzy, strojového učení, umělé inteligence, vědeckého výzkumu a mnoho dalších oblastí. Je to jeden z nejpopulárnějších programovacích jazyků na světě a má silnou pozici v oblasti vývoje softwaru.
    
    \begin{figure}[h]
    \centering
    \includegraphics[width=0.1\textwidth]{Python.png}
    \caption{Logo Pythonu}
    \end{figure}
   
\end{itemize}
\section{Historie Pythonu}
\begin{itemize}
    \item Python byl vytvořen v roce 1990-1991 Guido van Rossumem v Matematickém centru Stitching v Nizozemsku jako následník jazyka nazývaného ABC. Jméno Python pochází z Monty Python's Flying Circus. Tvůrci tohoto zábavného pořadu tvrdí, že nemá žádný význam a jim akorát připadalo zábavné. Nemá oficiální logo, přestože na mnoha webech se objevuje had (přesněji krajta), což je jeden z anglických významů slova Python. Přesto oba tvůrci souvislosti s hady odmítají. Python jazyk se vyvíjí a postupem času vznikly tři nekompatibilní major verze. Verze Python 1, která byla vydána v roce 1991 a jejíž poslední aktualizace proběhla v roce 2000. Nyní se tedy již nepoužívá. Další verze Python 2, která byla vydána v roce 2000 a jejíž poslední verze byla vydána v roce 2020. Python 3, verze 3.0 byla vydána v roce 2008 a je nyní aktuální řadou.
\end{itemize}
\begin{figure}[h]
    \centering
    \includegraphics[width=0.5\textwidth]{guido.jpg}
    \caption{Guido van Rossumem}
    \end{figure}
\begin{table}[htbp]
\centering
\caption{Tabulka vývojových milníků}
\label{tab:python-history}
\begin{tabular}{|c|p{7cm}|}
\hline
\textbf{Rok} & \textbf{Událost} \\
\hline
1989 &  Guido van Rossum začal vyvíjet Python v prosinci 1989. \\
\hline
1991 & První verze Pythonu 0.9.0. Tato verze obsahovala několik základních prvků, jako jsou blokové struktury a dynamické datové typy. \\
\hline
1994 & Python 1.0 byl vydán. Tato verze přinesla několik vylepšení a rozšíření, včetně mapování slovníku a modulu pro manipulaci s textovými řetězci. \\
\hline
2000 & Python 2.0, přinesl důležité změny a vylepšení, včetně podpory pro Unicode, list comprehensions a garbage collection. \\
\hline
2008 & Python 3.0, tato verze přinesla řadu zásadních změn, včetně sjednocení typů textu a bajtů, odstranění některých zastaralých funkcí a zlepšení syntaxe. \\
\hline
2010 & Python 2.7 byl vydán jako poslední verze série 2.x. Měl velkou bázi uživatelů a byl používán v mnoha projektech a aplikacích. \\
\hline
2015 & Python 3.5 přidal podporu pro await/async, nové operátory a typové kontroly. \\
\hline
2020 & Python 2.7 dosáhl konce životnosti a byl oficiálně ukončen. \\
\hline
2020 & Python 3.9 přinesl nové nástroje a zvýšil efektivitu vývoje v Pythonu. Umožnily programátorům psát čitelnější a efektivnější kód. \\
\hline
2023 & Aktuální datum - Python je stále populární a aktivně vyvíjen. \\
\hline
\end{tabular}
\end{table}
\section{Instalace Pythonu na Windows}
\begin{itemize}
    \item Běž na oficiální stránku pythonu \href{https://www.python.org/}{python.org}
    \item klikni na v menu na Download
    \item Stáhni si vhodný instalační balíček pro tvoji verzi windows (32bit/64bit).

    \item Stažený balíček spusť.Na začátku instalace zaškrtni:
            \textbf{Use admin privileges when installing py.exe} a také \textbf{add python.exe to PATH.} Tyto volby ti zjednoduší vytvoření virtuálního prostředí.
    \item Pak zmáčkni \textbf{Install now} a dále se drž instrukcí.

\end{itemize}
\begin{figure}[h]
    \centering
    \includegraphics[width=1\textwidth]{install.png}
    \caption{Instalační balíček pythonu}
    \label{fig:enter-label}
\end{figure}
\section{Python Enhancement Proposal}
\begin{itemize}
    \item Python Enhancement Proposal (PEP) je návrh nebo dokument, který slouží k popisu nových funkcí, rozšíření nebo změn v programovacím jazyku Python. PEPy jsou způsobem, jakým komunita Pythonu diskutuje o nápadech, které by měly být začleněny do jazyka, a jakým způsobem se tyto nápady stávají součástí oficiálního Pythonu.

    \item PEP 8 je jedním z Python Enhancement Proposal (PEP) dokumentů, který definuje oficiální směrnice pro styl psaní kódu v jazyce Python. Jeho název je "Style Guide for Python Code" a slouží k tomu, aby vývojáři psali konzistentní, čitelný a čistý kód, který je snadný k udržování a sdílení.

\item Zde je ukázka správně formátovaného bloku python kódu \textbf{listings}:
\begin{center}

\begin{lstlisting}
def pozdrav(name):
    if name:
        print("Ahoj, " + name)
    else:
        print("Ahoj, svete")
\end{lstlisting}
\end{center}
\item Zde je ukázka špatně formátovaného bloku python kódu \textbf{listings}:

\begin{lstlisting}
def pozdrav(name):
    if name:
    print("Ahoj, " + name)
    else:
        print("Ahoj, svete")
\end{lstlisting}
\item První ukázka obsahuje správně odsazený kód, zatímco druhá ukázka má nesprávné odsazení, což by bylo v rozporu s PEP 8.
\end{itemize}
\section{Klíčová slova Pythonu}
\begin{itemize}
    \item \textbf{and} - logický operátor: současně platí, a zároveň
    \item \textbf{as} - příkaz využitý při řešení výjimek
    \item \textbf{assert} - příkaz využitý při řešení výjimek
    \item \textbf{break} - ukončení současné činnosti a přechod na další (například vyskočení z cyklu)
    \item \textbf{class} - třída
    \item \textbf{continue} - ukončí současný průběh cyklu a zahájí nový průchod cyklem
    \item \textbf{def} - vytvoření vlastní funkce
    \item \textbf{del} - zrušení proměnné nebo funkce
    \item \textbf{elif} - příkaz při vícenásobném větvení (například: if ... elif ... elif ... else...)
    \item \textbf{else} - ukončení větvení - provede se to, co nesplňuje podmínky předchozích větví
    \item \textbf{except} - provede následující příkazy, jestliže dojde k výjimce (například: try ... except ...)
    \item \textbf{exec} - dynamické spouštění kódu programu
    \item \textbf{finally} - závěrečná část bloku, kde může nastat chyba (například: try ... except ... finally ...)
    \item \textbf{for} - cyklus se známým počtem opakování
    \item \textbf{from} - import funkce z externího modulu (například: from ... import ...)
    \item \textbf{global} - vytvoření globální proměnné, ke které lze přistupovat ze všech míst programu
    \item \textbf{if} - jestliže, zahajovací podmínka pro větvení
    \item \textbf{import} - načtení externího modulu nebo jeho položek
    \item \textbf{in} - logický operátor: v, ve = vrací True, jestliže to, co je vlevo, je obsaženo v tom, co je vpravo
    \item \textbf{is} - logický operátor: zjištění, zda jsou dva objekty totožné, obsahují stejná data
    \item \textbf{lambda} - vytvoření funkce pro jednorázové použití
    \item \textbf{local} - vytvoření lokální proměnné, například uvnitř funkce
    \item \textbf{nonlocal} - vytvoření proměnné s omezeným rozsahem platnosti
    \item \textbf{not} - logický operátor: není, opak, negace
    \item \textbf{or} - logický operátor: nebo
    \item \textbf{pass} - příkaz \textit{pass} nedělá nic, používá se tam, kde nějaký příkaz musí být, ale neprovádí se žádná činnost
    \item \textbf{raise} - příkaz využitý při řešení výjimek
    \item \textbf{return} - vrací výsledek funkce, podobně jako \textit{print}, ale bez jakýchkoliv úprav
    \item \textbf{try} - spouští blok, ve kterém se může očekávat chyba, výjimka (například: try ... except ... finally ...)
    \item \textbf{while} - cyklus s předem neznámým počtem opakování, provádí se, dokud platí podmínka
    \item \textbf{with} - používá se v kontextových managers
    \item \textbf{yield} - používá se v generátorech pro generování hodnot
    \item \textbf{True} - pravda, platí, 1, ano
    \item \textbf{False} - nepravda, neplatí, 0, ne
    \item \textbf{None} - žádný, nic
\end{itemize}
\section{Tabulka základních datových typů v Pythonu}
\begin{table}[htbp]
\centering
\caption{Základní/primitivní datové typy v Pythonu}
\label{tab:python-datatypes}
\begin{tabularx}{\textwidth}{|c|X|X|}
\hline
\textbf{Datový typ} & \textbf{Zastupuje} & \textbf{Příklady} \\
\hline
\texttt{int} & Celé číslo & 42, -5, 2 \\
\hline
\texttt{float} & Desetinné číslo & 3.14, -0.2, 1.2 \\
\hline
\texttt{bool} & Booleovská hodnota & \texttt{True}, \texttt{False} \\
\hline
\texttt{complex} & Komplexní číslo & $1 + 2d$, $-3 + 2n$, $0 + 10k$ \\
\hline
\texttt{str} & Řetězec znaků & 'Hello', '42' \\
\hline
\end{tabularx}
\end{table}
\section{Komentáře v Pythonu}
\begin{itemize}
\item \textbf{Jednořádkový komentář:} Jednořádkový komentář v Pythonu je způsob, jak vložit poznámku nebo vysvětlení přímo do kódu na jediném řádku. Komentáře jsou ignorovány interpretrem Pythonu a slouží pouze pro lidské čtení kódu. Syntaxe jednořádkového komentáře je následující:

\textbf{Příklad:}
\begin{lstlisting}
    # Toto je jednoradkovy komentar
\end{lstlisting}
\item \textbf{Víceřádkový komentář:} V Pythonu lze použít víceřádkové komentáře pomocí tzv. "docstringů" (dokumentačních řetězců), což jsou víceřádkové řetězce, které jsou umístěny na začátku funkce, třídy, modulu nebo metody a slouží k dokumentaci daného kódu. Docstrings jsou běžně používány pro vytváření dokumentace kódu, protože jsou snadno dostupné pomocí některých nástrojů a můžete je snadno extrahovat pro generování dokumentace.

\textbf{Příklad:}

""
    Toto je víceřádkový komentář,

    který slouží k dokumentaci této funkce.
    ""

\end{itemize}
\section{Použité Zdroje}
\begin{itemize}
  \item \href{https://kitner.cz/testovani_softwaru/jak-nainstalovat-python/}{1. Instalace Pythonu}
  \item \href{https://quizlet.com/503181776/python-klicova-slova-flash-cards/}{2. Klíčová slova Pythonu}
  \item \href{https://training.gismentors.eu/geopython-zacatecnik/python/comments.html}{3. Komentáře v Pythonu}
  \item \href{https://en.wikipedia.org/wiki/Python_(programming_language)}{4. Historie Pythonu}
\end{itemize}
\end{document}